\documentclass[10pt, preprint]{aastex}
\usepackage{fullpage,enumitem,amsmath,amssymb,graphicx}
\usepackage{graphicx} % This is a package for including graphics in your solution.
\usepackage{tikz}
\usepackage{listings}
\usetikzlibrary{automata, positioning, arrows}
\DeclareMathOperator*{\argmax}{arg\,max}

\tikzset{
	->,
	>=stealth'
}


\begin{document}

\begin{center}
{\Large CS 6466 2Chainz}


\begin{tabular}{lr}
Email: & wdaviau@protocol.ai \\
CUNet ID: & wtd37 \\
Name: & Wyatt Daviau \\
Date: & March 25th 2018 \\
\end{tabular}
\end{center}



\section{2Chainz -- Miner Dynamics Across Blockchains}

Items wrapped in * * need significant further thought
Big goals of paper:
\begin{enumerate}
\item
Exploring fundamentals
\begin{enumerate}
\item
Define the model we used, communicate its strengths and shortcomings
\item
Come up with a reasonable "greedy" mercenary miner strategy. Justify why miners would want to run this strategy and how it works.
\item
Explore the strategies implications from first principles.  Plot expected unstable regions.  Present a diagram of the four relevant quantities and how their ordering and difficulty adjustment creates sustained oscillations of hash power between the chains
\item
Describe the simulation framework and methodology.  Present simulation results.  Show the empirical breakdown of the parameter space into regions.  Quantify how the smaller chain's hash power distribution changes over the space.  Quantify the profitability of the scheme as compared to staying loyal to the first chain.
\end{enumerate}
\item
Extending Miner strategies
\begin{enumerate}
\item
*specify a strategy and discuss whether it is optimal*
\item
*Compare to greedy with simulations*
\end{enumerate}
\item
 Blockchain Protocol Gaming Greedy Mercenary Miners
 \begin{enumerate}
 \item
Define slight updates to the model.  The protocol is trying to optimize for hashrate on the chain and reducing the number of coins minted compared to the other chain.
\item
Describe the update protocol and motivation for it coming from the diagrams of A.2
\item
*Do some analysis of the optimality of this protocol, the gained hashrate for a target minting threshold as a function of parameters*
\item
*Simulations that run this protocol and record the minting savings and hashrate purchase*
\end{enumerate}

\item
D. Are miners following the mercenary strategy when they could be?
1. plot of thresholds over time highlighting regions where the miners could have switched.  Point out and places where it looks like they do switch
\end{enumerate}

\section{Abstract}
We model the scenario where two blockchains have the same proof of work hash function and therefore miners can switch among chains at will.  We devise a principled simple strategy for choosing among competing blockchains that miners can follow.  We demonstrate that this strategy is profitable throughout a large region of the considered parameter space and classify different regions of the parameter space based on the behavior of miners following this profitable strategy.  We present an analysis of historical BTC / BCH data in light of our 

Furthermore we investigate what an optimal switching strategy would look like if miners can make reasonable predictions about the future.  We also propose a candidate strategy for a blockchain designed to compete with an existing chain for hash power without flooding the market for its native token by minting coins at an excessive rate.

\section{Introduction}
The recent Bitcoin Cash (BCH) fork of the Bitcoin (BTC) blockchain serves as an example of a relatively powerful fork of a blockchain network.  Such forks naturally lead to multiple blockchain networks using the same proof of work function.  In such a world "mercenary" miners can potentially increase profits by dynamically switching their hash power across chains.  Recent events have shown that this can have detrimental effects on the networks involved [1] due to drastic changes in profitability among chains and a slow response to large changes in hashrate that trace back to details in bitcoin's difficulty adjustment algorithm.

The goal of our analysis is to study the scenario in which miners are willing to switch between two blockchains that run the same difficulty adjustment protocol to maximize their net profits.  We define a plausible parametrization of this system and then study potential chain-switching strategies that miners might employ in an attempt to increase their profits.  We begin by trying to understand the profitability over time of what we deemed the "greedy mercenary" approach, and the regions in which this leads to stable and unstable hash power allocations across chains.  From first principles we come up with a simple metric mercenary miners could use to decide on the best chain and end up rediscovering the so-called "Difficult Adjustment Reward Index" or DARI that miners actually use during operation. 

[Ahaan discussion of further optimal algorithms or optimality of greedy mercenary]

We model the system consisting of two chains both running BTC's difficulty adjustment protocol but sweeping different reward fractions, "loyal" hashrates and mercenary hashrate.  Based on our model we introduce a simple visualization for the stability characteristics of different points in the parameter space and an associated prediction function determining whether a point in the parameter space will be unstable. We built simple simulation and analysis frameworks for generating and interpreting data representing execution of these systems in the presence of mercenary miners following the greedy strategy.  Leveraging these frameworks we present a view into the different stability regions of the parameter space and reach the conclusion that the greedy mercenary mining strategy is profitable throughout a wide section of the parameter space.  We confirm our theoretical prediction of the instability region concluding that in the BTC / BTC difficulty adjustment case the region of instability in the parameter space is significant and  [impacts the transaction throughput of both systems TODO analysis function and figure quantifying this]. 

[ Arnesh + Wyatt TODO BCH / BTC, BCH / BCH simulation results ] 

[ Arnesh + Ahaan analysis of historical data sets ]


\section{Model and Assumptions}
We make many simplifying assumptions when modeling this system.  The goal is to make enough assumptions to keep analysis tractable while at the same time parameterizing the problem in a way that gives insight to real systems of competing blockchains, and paves the way for more advanced modeling.  The following are global system assumptions 
\begin{enumerate}
\item
There are two chains in the system, chain 1 and chain 2.  Both chains use the same proof of work and difficulty adjustment algorithm as the Bitcoin protocol [TODO when BCH difficulty is added in this will change]
\item
Each chain rewards miners with a coinbase transaction that pays out tokens to the miner that solves the block puzzle.  Additionally miners are rewarded by the transaction fees tied to the transactions serialized in the blocks.  We assume that the value of the total reward (in, say USD) of chain 1 is a constant $f_1$ and the value of the token of chain 2 is a constant $f_2$.  We do not account for variations in transaction fees or reductions in the coinbase payout in this analysis.  Only the ratio of these two parameters $\frac{f1}{f2}$ is relevant in our analysis.
\item
We model the total hash rate of the system as $H$ (hashes / second) and assume that it does not change over the period of analysis in question.  
\item
Fractions of the hashpower belong to three different entities, $\alpha$, $\beta_1$, $\beta_2$.  $\alpha$ + $\beta_1$ + $\beta_2$ = 1.
\item
We assume a fraction $\alpha$ of the hash power, from here on out called the mercenary miners, is willing to switch between chains in order to make a profit.  In the case of our analysis of the Greedy Mercenary Mining strategy we do NOT need to make the assumption that the hashpower $\alpha$ belongs to a single pool.  This is elaborated further in the following section.
\item
$\beta_1$ and $\beta_2$ corresponding to miners who are loyal to chains 1 and 2 respectively.  These hashpower fractions do not move from their respective chains regardless of the profitability of switching.  We need not consider the hash power of $\beta_i$ as belonging to a single pool.
\item
The difficulty of each chain is adjusted every $b$ blocks.  The time it takes for $b$ blocks to be mined in epoch $i$ = $\Delta_i$.  To calculate the difficulty adjustment the protocol specifies that: $d_{i+1} = \frac{ d_i \cdot b \cdot t_{\text{target}} }{ \Delta_i}$, as in bitcoin.  The target is 600s.

\item
We assume the system begins in a steady state where the difficulty of each chain reflects the initial allocation of the hash power.  The mercenary miners begin by mining on chain 1.  Throughout the analysis chain 1 is given the asymetric advantage that $\frac{f1}{f2} \leq 1$
\end{enumerate}

\subsection{Calculating the initial difficulty}

From assumption (8) above we can calculate the initial difficulty, which is an initial condition needed by the simulation framework.  We include the derivation here because the discussion is relevant for the subsequent derivation of the switching criterion for the greedy mining strategy.  

The following discussion applies to both chains and examines chain 1 without loss of generality.  By our steady state assumption that the mining epoch before the period of history in which our analysis takes place lasted exactly $(600 s)(b)$.  The number of hashes tried until a block is successfully mined follows a geometric distribution and by the properties of this distribution the inverse of the probability of one success is equal to the mean number of hashes until success.  Given our steady state condition and initial allocation we have the following relation
$$
600 H(\alpha+ \beta_1) = \frac{1}{p}
$$
Where $p$ is the probability of one hash solving the block puzzle.
Recall that the difficulty of a chain is given by
$$
\frac{F_{\text{max}}}{F}
$$
where $F$ is the threshold value below which a block with the given hash solves the puzzle.  As the range of the hash function used in the chain puzzle is binary strings of 256 characters, $p = \frac{F}{2^{256}}$ and therefore
\begin{align*}
600 H(\alpha + \beta_1) &= \frac{2^{256}}{F}\\
600 H(\alpha + \beta_1)F_{\text{max}} &= \frac{2^{256}F_{\text{max}}}{F}\\
\frac{600 H(\alpha + \beta_1)F_{\text{max}} }{2^{256}} &= d_{10}
\end{align*}
where $d_{1i}$ is the difficulty of chain 1 in the $i$th adjustment period of chain 1.  Factoring out constants we have
\begin{align*}
\mathcal{D} &= \frac{HF_{\text{max}} }{2^{256}}\\
d_{10} &= 600\mathcal{D} (\alpha + \beta_1)\\
d_{20} &= 600\mathcal{D} (\beta_2)\\
\end{align*}

\section{The Greedy strategy}
What happens if the switching miner decides to maximize for profit with a simple greedy algorithm, that is, an algorithm making no predictions about the future hash allocations or reward schedules of the two chains?  The sensible way to do this is to compare the expected rate of reward on both chains at any time.  We now show that the simplified form of the expected reward rate at difficulty adjustment period $i$ is simply $$\frac{f_1}{d_i} \cdot \alpha$$.  Let $\mathcal{E}_{ij}[T]$ be the expected time to mine a block on chain $j$ in difficulty period $i$.  The expectation is taken over the randomness in solving hash puzzles while assuming the difficulty $d_i$ is known.  To start we assume that there is only one mercenary mining pool with hash power $\alpha$, and later show how to relax this assumption.

As we saw in the last section the expected time to find a block can be related to the chain difficulty:
\begin{align*}
& \mathcal{E}_i[T] (\alpha + \beta_j)H = \frac{1}{p} \\
& \mathcal{E}_i[T] (\alpha + \beta_j)H = \frac{2^{256}}{F_{i}}\\
& \mathcal{E}_i[T] (\alpha + \beta_j)H = d_{ij} \frac{2^{256}}{F_{\text{max}}}\\
& \mathcal{E}_i[T] = \frac{d_{ij}}{\mathcal{D}(\alpha + \beta_j)} \\
\end{align*}

Now we can calculate the expected profit to mine on chain $j$ during difficulty period $i$ as follows
\begin{align*}
& \frac{\text{reward per block} }{\text{expected time per block}} \cdot (\text{expected winning fraction}) \\
&= \frac{f_j}{\mathcal{E}_i[T]} \cdot \frac{\alpha}{\alpha + \beta_j}\\
&= \frac{f_j \cdot \mathcal{D}(\alpha + \beta_j)}{d_{ij}} \cdot \frac{\alpha}{\alpha + \beta_j}\\
&= \frac{f_j \cdot \mathcal{D}\alpha}{d_{ij}}\\
\end{align*}

As $\mathcal{D}$ and $\alpha$ appear in the profit rates of both chains, the greedy mercenary mining strategy checks the difficulty of the current period on each chain $d_1$ and $d_2$ and mines on chain
$$\argmax{\frac{f_j}{d_j}}$$

In our simplifying model the fractional reward does not change and therefore the miner will only benefit from switching when one of the chains adjusts its difficulty.  When one profit rate is strictly greater than the other the mercenary miner does not stand to benefit in the short term from allocating some hashpower to the less profitable chain so all $\alpha$ goes to the most profitable chain.  In the case both profit rates are equal the greedy mercenary miner chooses a chain at random. 

A key observation regarding the profit rate is that while changes in the block reward and difficulty values influence which chain is most profitable, the addition or subtraction of other mercenary hash power within the same period does not affect the outcome on expectation.  Intuitively this is because the expected rate of block emission changes exactly inversely proportionally to the change in the expected winning fraction of a given mining pool.  This is encoded in the algebra above by the appearance of $(\alpha + \beta_j)$ in both the numerator and denominator.  As a consequence the choices of one greedy mercenary miner is not affected by the choices of another, at least in our simplifying fixed reward model.  This means that we can model $\alpha$ as the hash power of ALL mercenary miners, and that our results are easier to apply to this more general case.

\subsubsection{Theoretical Analysis}


\section{Simulations}

\subsection{Methodology}


\subsection{Results}


\section{Other Strategies}
[TODO Ahaan]


\section{Optimal Protocol}

\section{Historical Analysis of Data}
[TODO Arnesh]


%The next switching point is influenced by the miner's choice of which chain to mine on.  Call 

\subsection{References}
[1] https://bitcoinandtheblockchain.blogspot.co.uk/2017/08/chain-death-spiral-fatal-bitcoin.html?m=1



\end{document}